\documentclass{scrartcl}

% Adapted from an original template by Hlyni Arnórssyni, Reykjavik University, Iceland
%
% ------------------------------ SETTINGS
\usepackage{geometry}

\geometry{
    paper=a4paper, % Paper size
    top=2.5cm, % Top margin
    bottom=2.5cm, % Bottom margin
    left=2.5cm, % Left margin
    right=2.4cm, % Right margin
    headheight=0.75cm, % Header height
    footskip=1.5cm, % Space from the bottom margin to the baseline of the footer
    headsep=0.75cm, % Space from the top margin to the baseline of the header
    %showframe, % Uncomment to show how the type block is set on the page
}

\usepackage{blindtext}
%-------------------------------- Character encoding ----------------------------
\usepackage[T1]{fontenc}
\usepackage[utf8]{inputenc}

%----------------------------- Mathematics packages from AMS ---------------

\usepackage{amsmath, amsfonts, amsthm, amssymb}
\usepackage{braket, nicefrac}

% ----------- International System of Units
\usepackage{siunitx}

%------------------------------ Lists / numbers -------------------------
\usepackage{enumitem, multicol}

%------------------------------- Figure insertions --------------
\usepackage{graphicx, float}  % Use option [H] to force the placement of a figure
\usepackage{keystroke}
\usepackage{pgfplots}\usepgfplotslibrary{units}\pgfplotsset{compat=1.16}

%------------------------------- Line Spacing --------------
\usepackage{setspace}

%------------------------------- Depth of the ToC --------------
\setcounter{tocdepth}{2}

%%%%%%%%%%%%%%%%%%%%%%%%%% Hyperlink References %%%%%%%%%%%%%%%%%%%%%%%%%%%
\usepackage{hyperref}

\usepackage{listings}
\usepackage{textcomp}
\lstset{upquote=true}

%--------------------% Storage Path for images %-----------------%
\graphicspath{{graphics/}{Graphics/}{./}}

%%%%%%%%%%%%%%%%%%%%%%%%%% Environments %%%%%%%%%%%%%%%%%%%%%%%%%%%
\renewenvironment{abstract}{
    \begin{center}
        \textbf{Abstract}
        \vspace{0.5cm}
        \par\itshape
        \begin{minipage}{0.8\linewidth}}{\end{minipage}
        \noindent\ignorespaces
    \end{center}
}

\newenvironment{keywords}{
    \begin{center}
        \textbf{Keywords}
        \vspace{0.5cm}
        \par
        \begin{minipage}{0.8\linewidth}}{\end{minipage}
        \noindent\ignorespaces
    \end{center}
}

\newenvironment{preface}{
    \begin{center}
        \textbf{Preface}
        \vspace{0.5cm}
        \par
        \begin{minipage}{0.8\linewidth}}{\end{minipage}
        \noindent\ignorespaces
    \end{center}
}

\newenvironment{acknowledgements}{
    \begin{center}
        \textbf{Acknowledgements}
        \vspace{0.5cm}
        \par
        \begin{minipage}{0.8\linewidth}}{\end{minipage}
        \noindent\ignorespaces
    \end{center}
}

\usepackage{spalign}

\newcommand{\code}[1]{\texttt{#1}}

\begin{document}
%Title of the report, name of coworkers and dates (of experiment and of report).
\begin{titlepage}
    \centering
    {\large \today\par}
    \vfill

    %%%% PROJECT TITLE
    {\huge\bfseries Differential Swerve State Space Controller Design\par}
    \vfill

    %%%% AUTHOR(S)
    {\Large\itshape Henry LeCompte}\par
    {\itshape FRC Team 2383, The Ninjineers}\par
    \vspace{1.5cm}

    \vfill
    % Bottom of the page
\end{titlepage}

\newpage

\doublespacing
\tableofcontents
\singlespacing

\newpage

\doublespacing

\section{Introduction}
Swerve is a type of drivetrain that is characterized by having at least two wheels that can each rotate and spin independently of each other (here rotate is define as a rotation around the robots up axis and spin is defined as a rotation around the modules y axis). This allows the robot to move in any direction and turn in place. Differential Swerve is a type of swerve drivetrain in which the rotation of the wheel is controlled by the difference of the motor velocities and the spin if the wheel is controlled by the average of the motor velocities. This allows for the combination of the torque produced by both motors when driving and turning. This is the main advantage over conventional (coaxial) swerve drivetrains in which rotation and spin are controlled by separate motors. The main disadvantage of differential swerve is that it is significantly more complex to control.

\section{Designing the State Space Model}

State space controller in the form of
\begin{equation} \label{state-change}
    \dot{x} = Ax + Bu
\end{equation}
\begin{equation} \label{output}
    y = Cx + Du
\end{equation}

\subsection{Inputs, Outputs, and States}
The States will be in the form of
\begin{equation}
    x =
    \begin{bmatrix}
        v_t    \\
        v_b    \\
        \theta \\
    \end{bmatrix}
\end{equation}

Where \(\theta\) is the angle of the wheel and\(v_t\) and \(v_b\) are the top and bottom motor velocities respectively. \\
The velocity of the wheel is not included as a state because it is a linear combination of the motor velocities so it can be calculated from the current state.

The inputs will be in the form of
\begin{equation}
    u =
    \begin{bmatrix}
        V_t \\
        V_b \\
    \end{bmatrix}
\end{equation}

Where \(V_t\) is the top motor voltage and \(V_b\) is the bottom motor voltage.

Our output matrix is the same as the state matrix as we can measure all states.
\begin{equation}
    y =
    \begin{bmatrix}
        v_t \\
        v_b \\
        \theta
    \end{bmatrix}
\end{equation}

Where \(v_t\) is the top motor velocity, \(v_b\) is the bottom motor velocity, and \(\theta\) is the angle of the wheel.

\subsection{DC Motor Model}
We know that a permanent magnet DC motor follows the general equation of
\begin{equation}
    V = K_v\dot{x} + K_a\ddot{x}
\end{equation}
And we can rewrite this as
\begin{align*}
    V               & = K_v\dot{x} + K_a\ddot{x}                \\
    V - K_a\ddot{x} & = K_v\dot{x}                              \\
    -K_a\ddot{x}    & = K_v\dot{x} - V                          \\
    \ddot{x}        & = \frac{-K_v\dot{x} + V}{K_a}             \\
    \ddot{x}        & = \frac{-K_v\dot{x}}{K_a} + \frac{V}{K_a}
\end{align*}
We can also substitute v as \(\dot{x}\) to create
\begin{equation}
    \dot{v} = \frac{-K_v v}{K_a} + \frac{V}{K_a}
\end{equation}

This equation can then be written in state space form as
\begin{equation}
    \dot{x} = \begin{bmatrix}\frac{-K_v}{K_a}\end{bmatrix}x + \begin{bmatrix}\frac{1}{K_a}\end{bmatrix}u
\end{equation}
\begin{equation}
    y = 1x + 0u
\end{equation}

\subsection{A and B Matrices}
Now that we know how to calculate the angular velocity of a motor based on its constants and the input voltage we can start to create the formulas needed to compute the different velocity components of the system
\begin{align}
    \dot{v_t}    & = \frac{-K_{v_\mathit{drive}}}{K_{a_\mathit{drive}}}v_t + \frac{1}{K_{a_\mathit{drive}}}V_t \\
    \dot{v_b}    & = \frac{-K_{v_\mathit{drive}}}{K_{a_\mathit{drive}}}v_b + \frac{1}{K_{a_\mathit{drive}}}V_b \\
    \dot{\theta} & = \frac{v_t + v_b}{2} * K_\mathit{turn\_ratio}
\end{align}

We can now rewrite these equations as a system that has coefficients for every state and input
\begin{align}
    \dot{v_t}    & = \frac{-K_{v_\mathit{drive}}}{K_{a_\mathit{drive}}}v_t + 0v_b + 0\theta + \frac{1}{K_{a_\mathit{drive}}}V_t + 0V_b                                                                                  \\
    \dot{v_b}    & = 0v_t + \frac{-K_{v_\mathit{drive}}}{K_{a_\mathit{drive}}}v_b + 0\theta + 0V_t + \frac{1}{K_{a_\mathit{drive}}}V_b                                                                                  \\
    \dot{v_w}    & = -\frac{K_{v_\mathit{drive}}}{2K_{a_\mathit{drive}}}v_t + \frac{K_{v_\mathit{drive}}}{2K_{a_\mathit{drive}}}v_b + 0\theta + \frac{1}{2K_{a_\mathit{drive}}}V_t - \frac{1}{2K_{a_\mathit{drive}}}V_b \\
    \dot{\theta} & = \frac{K_\mathit{turn\_ratio}}{2}v_t + \frac{K_\mathit{turn\_ratio}}{2}v_b + 0\theta + 0V_t + 0V_b
\end{align}

Now that we have these linear equations we can turn them into the A and B matrix

\begin{align}
    A & =
    \begin{bmatrix}
        \frac{-K_{v_\mathit{drive}}}{K_{a_\mathit{drive}}} & 0                                                  & 0 \\
        0                                                  & \frac{-K_{v_\mathit{drive}}}{K_{a_\mathit{drive}}} & 0 \\
        \frac{K_\mathit{turn\_ratio}}{2}                   & \frac{K_\mathit{turn\_ratio}}{2}                   & 0 \\
    \end{bmatrix} \\
    B & =
    \begin{bmatrix}
        \frac{1}{K_{a_\mathit{drive}}} & 0                              \\
        0                              & \frac{1}{K_{a_\mathit{drive}}} \\
        0                              & 0                              \\
    \end{bmatrix}
\end{align}

\subsection{C and D matrices}
With these, we just need the C and D matrices.
Because the output matrix is the same as the state matrix C is just the identity matrix.
\begin{align}
    C & =
    \begin{bmatrix}
        1 & 0 & 0 \\
        0 & 1 & 0 \\
        0 & 0 & 1 \\
    \end{bmatrix} \\
    D & = 0
\end{align}

\subsection{Full model}
And now will all of these matrices we can write the full continuos model as
\begin{equation}
    \dot{\begin{bmatrix}
            v_t    \\
            v_b    \\
            \theta \\
        \end{bmatrix}} =
    \begin{bmatrix}
        \frac{-K_{v_\mathit{drive}}}{K_{a_\mathit{drive}}} & 0                                                  & 0 \\
        0                                                  & \frac{-K_{v_\mathit{drive}}}{K_{a_\mathit{drive}}} & 0 \\
        \frac{K_\mathit{turn\_ratio}}{2}                   & \frac{K_\mathit{turn\_ratio}}{2}                   & 0 \\
    \end{bmatrix}\begin{bmatrix}
        v_t    \\
        v_b    \\
        \theta \\
    \end{bmatrix} + \begin{bmatrix}
        \frac{1}{K_{a_\mathit{drive}}} & 0                              \\
        0                              & \frac{1}{K_{a_\mathit{drive}}} \\
        0                              & 0                              \\
    \end{bmatrix}\begin{bmatrix}
        V_t \\
        V_b \\
    \end{bmatrix}
\end{equation}
\begin{equation}
    \begin{bmatrix}
        v_t    \\
        v_b    \\
        \theta \\
    \end{bmatrix}      = \begin{bmatrix}
        1 & 0 & 0 \\
        0 & 1 & 0 \\
        0 & 0 & 1 \\
    \end{bmatrix}\begin{bmatrix}
        v_t    \\
        v_b    \\
        \theta \\
    \end{bmatrix} + 0\begin{bmatrix}
        V_t \\
        V_b \\
    \end{bmatrix}
\end{equation}

\newpage

\section{Validating the State Space Model}
Now that we have a state space model

\newpage

\section{Designing the LQR}
The LQR is designed to minimize the cost function

\end{document}
