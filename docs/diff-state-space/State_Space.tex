\documentclass{scrartcl}

% Adapted from an original template by Hlyni Arnórssyni, Reykjavik University, Iceland
%
% ------------------------------ SETTINGS
\usepackage{geometry}

\geometry{
    paper=a4paper, % Paper size
    top=2.5cm, % Top margin
    bottom=2.5cm, % Bottom margin
    left=2.5cm, % Left margin
    right=2.4cm, % Right margin
    headheight=0.75cm, % Header height
    footskip=1.5cm, % Space from the bottom margin to the baseline of the footer
    headsep=0.75cm, % Space from the top margin to the baseline of the header
    %showframe, % Uncomment to show how the type block is set on the page
}

\usepackage{blindtext}
%-------------------------------- Character encoding ----------------------------
\usepackage[T1]{fontenc}
\usepackage[utf8]{inputenc}

%----------------------------- Mathematics packages from AMS ---------------

\usepackage{amsmath, amsfonts, amsthm, amssymb}
\usepackage{braket, nicefrac}

% ----------- International System of Units
\usepackage{siunitx}

%------------------------------ Lists / numbers -------------------------
\usepackage{enumitem, multicol}

%------------------------------- Figure insertions --------------
\usepackage{graphicx, float}  % Use option [H] to force the placement of a figure
\usepackage{keystroke}
\usepackage{pgfplots}\usepgfplotslibrary{units}\pgfplotsset{compat=1.16}

%------------------------------- Line Spacing --------------
\usepackage{setspace}

%------------------------------- Depth of the ToC --------------
\setcounter{tocdepth}{2}

%%%%%%%%%%%%%%%%%%%%%%%%%% Hyperlink References %%%%%%%%%%%%%%%%%%%%%%%%%%%
\usepackage{hyperref}

\usepackage{listings}
\usepackage{textcomp}
\lstset{upquote=true}

%--------------------% Storage Path for images %-----------------%
\graphicspath{{graphics/}{Graphics/}{./}}

%%%%%%%%%%%%%%%%%%%%%%%%%% Environments %%%%%%%%%%%%%%%%%%%%%%%%%%%
\renewenvironment{abstract}{
    \begin{center}
        \textbf{Abstract}
        \vspace{0.5cm}
        \par\itshape
        \begin{minipage}{0.8\linewidth}}{\end{minipage}
        \noindent\ignorespaces
    \end{center}
}

\newenvironment{keywords}{
    \begin{center}
        \textbf{Keywords}
        \vspace{0.5cm}
        \par
        \begin{minipage}{0.8\linewidth}}{\end{minipage}
        \noindent\ignorespaces
    \end{center}
}

\newenvironment{preface}{
    \begin{center}
        \textbf{Preface}
        \vspace{0.5cm}
        \par
        \begin{minipage}{0.8\linewidth}}{\end{minipage}
        \noindent\ignorespaces
    \end{center}
}

\newenvironment{acknowledgements}{
    \begin{center}
        \textbf{Acknowledgements}
        \vspace{0.5cm}
        \par
        \begin{minipage}{0.8\linewidth}}{\end{minipage}
        \noindent\ignorespaces
    \end{center}
}

\usepackage{spalign}

\newcommand{\code}[1]{\texttt{#1}}

\begin{document}
%Title of the report, name of coworkers and dates (of experiment and of report).
\begin{titlepage}
    \centering
    {\large \today\par}
    \vfill

    %%%% PROJECT TITLE
    {\huge\bfseries Differential Swerve State Space Controller Design\par}
    \vfill

    %%%% AUTHOR(S)
    {\Large\itshape Henry LeCompte}\par
    {\itshape FRC Team 2383, The Ninjineers}\par
    \vspace{1.5cm}

    \vfill
    % Bottom of the page
\end{titlepage}

\newpage

\doublespacing
\tableofcontents
\singlespacing

\newpage

\doublespacing

\section{Designing the State Space Model}

State space controller in the form of
\begin{equation} \label{state-change}
    \dot{x} = Ax + Bu
\end{equation}
\begin{equation} \label{output}
    y = Cx + Du
\end{equation}

\subsection{Inputs, Outputs, and States}
The States will be in the form of
\begin{equation}
    x =
    \begin{bmatrix}
        \dot{L} \\
        \dot{R} \\
        \theta  \\
    \end{bmatrix}
\end{equation}

The inputs will be in the form of
\begin{equation}
    u =
    \begin{bmatrix}
        V_l & V_r
    \end{bmatrix}
\end{equation}

And the outputs will be the same as the states so C will be the identity matrix and D will be 0

\subsection{A and B matrices}
We know that a permanent magnet DC motor follows the general equation of
\begin{equation}
    V = K_v\dot{x} + K_a\ddot{x}
\end{equation}
And we can rewrite this as
\begin{align*}
    V               & = K_v\dot{x} + K_a\ddot{x}                \\
    V - K_a\ddot{x} & = K_v\dot{x}                              \\
    -K_a\ddot{x}    & = K_v\dot{x} - V                          \\
    \ddot{x}        & = \frac{-K_v\dot{x} + V}{K_a}             \\
    \ddot{x}        & = \frac{-K_v\dot{x}}{K_a} + \frac{V}{K_a}
\end{align*}
We can also substitute v as \(\dot{x}\) to create
\begin{equation}
    \dot{v} = \frac{-K_v v}{K_a} + \frac{V}{K_a}
\end{equation}

This equation can then be written in state space form as
\begin{equation}
    \dot{x} = \begin{bmatrix}\frac{-K_v}{K_a}\end{bmatrix}x + \begin{bmatrix}\frac{1}{K_a}\end{bmatrix}u
\end{equation}
\begin{equation}
    y = 1x + 0u
\end{equation}

Now that we know how to calculate the angular velocity of a motor based on its constants and the input voltage we can start to create the formulas needed to compute the different velocity components of the system
\begin{align}
    \ddot{L}     & = \frac{-K_{v_\mathit{drive}}}{K_{a_\mathit{drive}}}\dot{L} + \frac{1}{K_{a_\mathit{drive}}}V_l \\
    \ddot{R}     & = \frac{-K_{v_\mathit{drive}}}{K_{a_\mathit{drive}}}\dot{R} + \frac{1}{K_{a_\mathit{drive}}}V_r \\
    \dot{\theta} & = \frac{\dot{L} + \dot{R}}{2} * K_\mathit{turn\_ratio}
\end{align}

Now that we have these basic equations we can turn them into the A and B matrix
\begin{align}
    A & =
    \begin{bmatrix}
        \frac{-K_{v_\mathit{drive}}}{K_{a_\mathit{drive}}} & 0                                                  & 0 \\
        0                                                  & \frac{-K_{v_\mathit{drive}}}{K_{a_\mathit{drive}}} & 0 \\
        \frac{K_\mathit{turn\_ratio}}{2}                   & \frac{K_\mathit{turn\_ratio}}{2}                   & 0
    \end{bmatrix} \\
    B & =
    \begin{bmatrix}
        \frac{1}{K_{a_\mathit{drive}}} & 0                              \\
        0                              & \frac{1}{K_{a_\mathit{drive}}} \\
        0                              & 0                              \\
    \end{bmatrix}
\end{align}

\subsection{C and D matrices}
With these, we just need the C and D matrices
\begin{align}
    C & =
    \begin{bmatrix}
        1 & 0 & 0 \\
        0 & 1 & 0 \\
        0 & 0 & 1 \\
    \end{bmatrix} \\
    D & = 0
\end{align}
Now that we have our full state space representation of the system we need to create a Linear Quadratic Regulator to control the system.

\section{Designing the LQR}
The LQR is designed to minimize the cost function

\end{document}
