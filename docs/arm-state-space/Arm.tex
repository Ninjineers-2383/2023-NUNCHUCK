\documentclass{scrartcl}

% Adapted from an original template by Hlyni Arnórssyni, Reykjavik University, Iceland
%
% ------------------------------ SETTINGS
\usepackage{geometry}

\geometry{
    paper=a4paper, % Paper size
    top=2.5cm, % Top margin
    bottom=2.5cm, % Bottom margin
    left=2.5cm, % Left margin
    right=2.4cm, % Right margin
    headheight=0.75cm, % Header height
    footskip=1.5cm, % Space from the bottom margin to the baseline of the footer
    headsep=0.75cm, % Space from the top margin to the baseline of the header
    %showframe, % Uncomment to show how the type block is set on the page
}

\usepackage{blindtext}
%-------------------------------- Character encoding ----------------------------
\usepackage[T1]{fontenc}
\usepackage[utf8]{inputenc}

%----------------------------- Mathematics packages from AMS ---------------

\usepackage{amsmath, amsfonts, amsthm, amssymb}
\usepackage{braket, nicefrac}

% ----------- International System of Units
\usepackage{siunitx}

%------------------------------ Lists / numbers -------------------------
\usepackage{enumitem, multicol}

%------------------------------- Figure insertions --------------
\usepackage{graphicx, float}  % Use option [H] to force the placement of a figure
\usepackage{keystroke}
\usepackage{pgfplots}\usepgfplotslibrary{units}\pgfplotsset{compat=1.16}

%------------------------------- Line Spacing --------------
\usepackage{setspace}

%------------------------------- Depth of the ToC --------------
\setcounter{tocdepth}{2}

%%%%%%%%%%%%%%%%%%%%%%%%%% Hyperlink References %%%%%%%%%%%%%%%%%%%%%%%%%%%
\usepackage{hyperref}

\usepackage{listings}
\usepackage{textcomp}
\lstset{upquote=true}

%--------------------% Storage Path for images %-----------------%
\graphicspath{{graphics/}{Graphics/}{./}}

%%%%%%%%%%%%%%%%%%%%%%%%%% Environments %%%%%%%%%%%%%%%%%%%%%%%%%%%
\renewenvironment{abstract}{
    \begin{center}
        \textbf{Abstract}
        \vspace{0.5cm}
        \par\itshape
        \begin{minipage}{0.8\linewidth}}{\end{minipage}
        \noindent\ignorespaces
    \end{center}
}

\newenvironment{keywords}{
    \begin{center}
        \textbf{Keywords}
        \vspace{0.5cm}
        \par
        \begin{minipage}{0.8\linewidth}}{\end{minipage}
        \noindent\ignorespaces
    \end{center}
}

\newenvironment{preface}{
    \begin{center}
        \textbf{Preface}
        \vspace{0.5cm}
        \par
        \begin{minipage}{0.8\linewidth}}{\end{minipage}
        \noindent\ignorespaces
    \end{center}
}

\newenvironment{acknowledgements}{
    \begin{center}
        \textbf{Acknowledgements}
        \vspace{0.5cm}
        \par
        \begin{minipage}{0.8\linewidth}}{\end{minipage}
        \noindent\ignorespaces
    \end{center}
}

\usepackage{spalign}

\newcommand{\code}[1]{\texttt{#1}}

\begin{document}
%Title of the report, name of coworkers and dates (of experiment and of report).
\begin{titlepage}
    \centering
    {\large \today\par}
    \vfill

    %%%% PROJECT TITLE
    {\huge\bfseries Double Pink Arm With Wrist Controller Design\par}
    \vfill

    %%%% AUTHOR(S)
    {\Large\itshape Javier Irizarry-Delgado}\par
    {\itshape FRC Team 2383, The Ninjineers}\par
    \vspace{1.5cm}

    \vfill
    % Bottom of the page
\end{titlepage}

\newpage

\doublespacing
\tableofcontents
\singlespacing
\newpage
\doublespacing
\section{Introduction}
\subsection{What is the Pink Arm?}
The Pink Arm is an arm design originally used by Team 233 (The Pink Team) in 2004 to play defense and climb in endgame. It is comprised of a single jointed arm, with a telescoping joint at the end. This is a very versatile design, as it allows a subsystem mounted at the end to span a large area. This is useful for manipulating game elements, and has proven over the years to be a very efficient design in FRC.

\subsection{Purpose}
The purpose of this paper is to explain how concepts like PID control, feedforwards, forward and inverse kinematics, motion profiling, and presets, play a role in the programming of a complex robot subsystem. These concepts can turn a mechanism into a competitively viable subsystem that is easy to control in the teleoperated and autonomous periods of the FRC match.

\subsection{Who should read this paper?}
This paper is aimed at high school students in FRC who are looking to learn how to control a complex robot subsystem and apply the principles in this paper to their own robots. However, these principles are not exclusive to FRC and can be applied in many other places in the real world.

\subsection{What is a double pink arm with a wrist?}
The double pink arm with a wrist is a spin on the original pink arm design, created by team 2383, the Ninjineers. It comprises of a single jointed arm, with a telescoping extension, and a wrist at the end. This allows for three degrees of freedom to score cones and cubes. Two neo's at the base control the pivot, two neo 550's control the telescoping extension arms, and one neo 550 controls the wrist at the very top.

\section{Kinematics}
\subsection{Forward Kinematics}
The three axes of the arm, will be denoted as following. 

\begin{equation} \label{Notation}
    \theta_m,
    \theta_{wrel},
    l
\end{equation}

Where \(\theta_m\) is the angle of the pivot joint, \(\theta_{wrel}\) is the angle of the wrist joint relative to the arm itself, and \(l\) is the length of the arm. These three axes can be mathematically converted into an x, y, and \(\theta_{wabs}\) coordinate system, where \(\theta_{wabs}\) is the absolute angle of the wrist joint with regards to the x axis. The x and y values are the coordinates that the tip of the arm fall on. The conversion is as follows: 

\begin{equation} \label{X Coordinate}
    x = l \cos(\theta_m)
\end{equation}

\begin{equation} \label{Y Coordinate}
    y = l \sin(\theta_m)
\end{equation}

\begin{equation} \label{Pivot Absolute Coordinate}
    \theta_{wabs} = \theta_m + \theta_{wrel}
\end{equation}

\subsection{Inverse Kinematics}
When controlling a mechanism like the pink arm, it is often more convenient to control the mechanism in terms of the x, y, and \(\theta_{wabs}\) coordinates, rather than the \(\theta_m\), \(\theta_{wrel}\), and \(l\) coordinates. This is because the x, y, and \(\theta_{wabs}\) coordinates are more intuitive to control, and are easier to visualize. It is much easier to measure values for the arm to reach and input them into the system rather than calculate the conversions by hand for every preset. However, the x, y, and \(\theta_{wabs}\) coordinates are not directly related to the \(\theta_m\), \(\theta_{wrel}\), and \(l\) coordinates. This is where inverse kinematics comes in. Inverse kinematics is the process of converting the x, y, and \(\theta_{wabs}\) coordinates into the \(\theta_m\), \(\theta_{wrel}\), and \(l\) coordinates. The inverse kinematics for the pink arm are as follows:

\begin{equation} \label{theta_m}
    \theta_m = \arctan\biggl(\frac{y}{x}\biggl)
\end{equation}

\begin{equation} \label{l}
    l = \sqrt{x^2 + y^2}
\end{equation}

\begin{equation} \label{theta_wrel}
    \theta_{wrel} = \theta_m - \theta_{wabs}
\end{equation}

\section{PID Control}

\section{Feedforwards}
\subsection{Kv and Ka}
\subsection{Ks}
\subsection{Kg}

\section{Motion Profiling}

\section{Presets}

\end{document}
